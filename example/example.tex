\documentclass[10pt, aspectratio=169]{oden_beamer}
% Try using aspectratio=169, handout, blue, and/or dark in the [] above.

%%%%%%%%%%%%%%%%%%%%%%
% package imports here
%%%%%%%%%%%%%%%%%%%%%%

% \setbeameroption{show notes}  % Print speaker notes

\AtBeginSection[]{              % Show outline at beginning of each section
\begin{frame}
    \frametitle{Outline}
    \tableofcontents[currentsection]
\end{frame}
}

\begin{document} % ============================================================

% Title page definitions ------------------------------------------------------
\title{My Computational Science Presentation}
\author{\textbf{Author/Presenter}
        \\
        Author/Presenter 2
        \\
        Author/Presenter 3}
% \institute{Oden Institute for Computational Engineering and Sciences}
\date{
    \centering
    % Conference Name
    % \\
    The Date
}

\begin{frame} % ---------------------------------------------------------------
\titlepage
\end{frame}

\section{How Basic Elements Look} % ===========================================

\begin{frame} % ---------------------------------------------------------------
\frametitle{Regular Titles Are Colored}
\framesubtitle{So are subtitles!}

This is what a typical paragraph of text looks like on the slides.
This is what a typical paragraph of text looks like on the slides.
This is what a typical paragraph of text looks like on the slides.

\vspace{.5cm}
This is what an equation looks like (Green's first identity).
\begin{equation*}
    \int_{\Omega} \nabla u \cdot \nabla v \:d\Omega
    = \int_{\Gamma} v \nabla u \cdot \hat{\mathbf{n}} \:d\Gamma
    - \int_{\Omega} v \Delta u \:d\Omega
\end{equation*}
Use the \texttt{itemize} environment for a bullet-point list:
\begin{itemize}
    \item First item
    \item Second item
    \item[$\ast$] Item with a custom marker
\end{itemize}

Use the \texttt{enumerate} environment for a numbered list:
\begin{enumerate}
    \item First item
    \item Second item
\end{enumerate}
\end{frame}

\begin{frame} % ---------------------------------------------------------------
\frametitle{This Slide Has Two Columns and Several Blocks}
\begin{columns}
\begin{column}{.49\textwidth}
    Slides only have one column by default.
    Use the \texttt{columns} environment to make two or more columns.
    Columns can take the full page, or just part of a page.
\end{column}
%
\begin{column}{.49\textwidth}
    \begin{block}{A Block in a Column}
        This is called a \texttt{block}.
        It makes important points stand out!
        Use blocks wisely.
    \end{block}
\end{column}
\end{columns}

\vspace{.5cm}
Now we're back to a single-column format.

\pause % "animation" command.

\begin{block}{A Default Block}
     This is the size of a default block in the default single-column format.
\end{block}

\begin{center}
\begin{minipage}{.9\textwidth}
\begin{block}{A non-quite-so-wide-but-still-centered Block}
    The size of a block can be controlled with \texttt{minipage}.
\end{block}
\end{minipage}
\end{center}
\end{frame}

% Speaker notes for the previous frame.
\note[itemize]{
    \item This is a note to remind the speaker what to do.
    \item Use \texttt{\textbackslash setbeameroption\{show notes\}} in the preamble to show the speaker notes.
}

\section{Oden Theme Colors} % =================================================

\begin{frame}
\frametitle{Oden Color Palette}
The \texttt{oden\_beamer} class defines the following colors:

\begin{center}
\begin{tabular}{lr}
    \texttt{oden\_blue} &
        \setbeamercolor{colorsample}{bg=oden_blue}
        \begin{beamercolorbox}[wd=2cm,ht=.5cm]{colorsample}
        \end{beamercolorbox}
    \\
    \texttt{oden\_blue\_dark} &
        \setbeamercolor{colorsample}{bg=oden_blue_dark}
        \begin{beamercolorbox}[wd=2cm,ht=.5cm]{colorsample}
        \end{beamercolorbox}
    \\
    \texttt{burnt\_orange} &
        \setbeamercolor{colorsample}{bg=burnt_orange}
        \begin{beamercolorbox}[wd=2cm,ht=.5cm]{colorsample}
        \end{beamercolorbox}
    \\
    \texttt{accent\_orange} &
        \setbeamercolor{colorsample}{bg=accent_orange}
        \begin{beamercolorbox}[wd=2cm,ht=.5cm]{colorsample}
        \end{beamercolorbox}
    \\
    \texttt{charcoal} &
        \setbeamercolor{colorsample}{bg=charcoal}
        \begin{beamercolorbox}[wd=2cm,ht=.5cm]{colorsample}
        \end{beamercolorbox}
    \\
    \texttt{chalk} &
        \setbeamercolor{colorsample}{bg=chalk}
        \begin{beamercolorbox}[wd=2cm,ht=.5cm]{colorsample}
        \end{beamercolorbox}
\end{tabular}
\end{center}
Change the text color with \texttt{\textbackslash textcolor\{colorname\}\{text here\}}.
\begin{center}
\begin{tabular}{lcr}
    \texttt{\textbackslash textcolor\{accent\_orange\}\{some text\}}
    & $\longrightarrow$ &
    \textcolor{accent_orange}{some text}
\end{tabular}
\end{center}
\end{frame}

\begin{frame} % ---------------------------------------------------------------
\frametitle{Color Themes}
The \texttt{oden\_beamer} class implements four main color themes.
Select the theme by adding arguments to the \texttt{\textbackslash documentclass} command at the beginning of the \texttt{.tex} file.
\begin{itemize} % The <1-> and <2-> indicators "animate" the bullet points.
    \item<2-> \textcolor{highlight}{\underline{Light Orange}} (the default): white background, charcoal text, orange highlights.
    Usage: \texttt{\textbackslash documentclass[10pt]\{oden\_beamer\}}
    \vspace{.1cm}
    \item<2-> \textcolor{highlight}{\underline{Light Blue}}: white background, charcoal text, blue highlights.
    Usage: \texttt{\textbackslash documentclass[10pt, \underline{blue}]\{oden\_beamer\}}
    \vspace{.1cm}
    \item<3-> \textcolor{highlight}{\underline{Dark Orange}}: charcoal background, chalk text, orange highlights.
    Usage: \texttt{\textbackslash documentclass[10pt, \underline{dark}]\{oden\_beamer\}}
    \vspace{.1cm}
    \item<3-> \textcolor{highlight}{\underline{Dark Blue}}: charcoal background, chalk text, blue highlights.
    Usage: \texttt{\textbackslash documentclass[10pt, \underline{dark}, \underline{blue}]\{oden\_beamer\}}
\end{itemize}

\vspace{.25cm}
\visible<4->{
The theme also defines the colors \texttt{text} and \texttt{highlight}.
Using these names instead of specific colors makes it easy to switch between themes.
\begin{center}
\begin{tabular}{lcr}
    \texttt{\textbackslash textcolor\{highlight\}\{some text\}}
    & $\longrightarrow$ &
    \textcolor{highlight}{some text}
\end{tabular}
\end{center}}
\end{frame}

\begin{frame} % ---------------------------------------------------------------
\frametitle{Color Theme Examples}

\begin{center}
\begin{tabular}{lr}
    Light Orange: &
        \setbeamercolor{colorsample}{bg=white,fg=charcoal}
        \begin{beamercolorbox}[wd=5cm,ht=.5cm,dp=.25cm,center]{colorsample}
            \textcolor{burnt_orange}{Title text} Regular text
        \end{beamercolorbox}
    \\ \\
    Light Blue: &
        \setbeamercolor{colorsample}{bg=white,fg=charcoal}
        \begin{beamercolorbox}[wd=5cm,ht=.5cm,dp=.25cm,center]{colorsample}
            \textcolor{oden_blue_dark}{Title text} Regular text
        \end{beamercolorbox}
    \\ \\
    Dark Orange: &
        \setbeamercolor{colorsample}{bg=charcoal,fg=chalk}
        \begin{beamercolorbox}[wd=5cm,ht=.5cm,dp=.25cm,center]{colorsample}
            \textcolor{accent_orange}{Title text} Regular text
        \end{beamercolorbox}
    \\ \\
    Dark Blue: &
        \setbeamercolor{colorsample}{bg=charcoal,fg=chalk}
        \begin{beamercolorbox}[wd=5cm,ht=.5cm,dp=.25cm,center]{colorsample}
            \textcolor{oden_blue}{Title text} Regular text
        \end{beamercolorbox}
    \\ \\
\end{tabular}
\end{center}

The dark themes are particularly effective for virtual presentations (Zoom, etc.) and for rooms with low lighting.
\end{frame}

\end{document}
